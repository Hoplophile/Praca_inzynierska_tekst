\chapter{Hardware}
\label{cha:hardware}

Największą częścią projektu było wykonanie hardware'u słuchawek. Dobór parametrów elementów akustycznych był szczególnie trudny, biorąc pod uwagę brak doświadczenia w tym obszarze elektroniki. Wszystkie elementy schematu musiały być wybrane pod kątem minimalizacji szumów i poboru mocy, a zaprojektowana płytka musiała się zmieścić do obudowy słuchawek i pozwolić na wyprowadzenie na zewnątrz mikrofonu, przycisków oraz gniazda ładowania. To wymagało przemyślanego wymiarowania zarówno modelu, jak i płytki oraz wyprowadzenia w odpowiedni sposób elementów, na przykład stosując kątowe przyciski.

Dla projektu obudowy głównym ograniczeniem było to, aby słuchawki były kompaktowe, czyli lekkie i niskoprofilowe. Ochronniki tego typu są przeznaczone dla myśliwych, strzelców sportowych, ale również dla służb ochrony i wojsk specjalnych. Toteż nie mogą ograniczać ruchów, możliwości przyłożenia głowy do kolby broni i być znaczącym ciężarem podczas użytkowania przez kilka godzin lub dni w terenie.

Ten sam powód determinuje wymóg niskiego poboru prądu przez układ i dużej pojemności akumulatora. Choć w tej pracy został wybrany wbudowany akumulator ładowany przez gniazdo mikro USB, to do zastosowań wojskowych lepsze byłoby zasilanie ze zwykłych, wymiennych baterii, na przykład AAA.

Układy scalone zastosowane w projekcie musiały mieć możliwość zasilania napięciem 3.3V, ponieważ zastosowany został akumulator litowo-jonowy o napięciu znamionowym $3,7V$ (zakres pracy wynosi od $3,0$ do $4,2V$).

Dla uproszczenia, podczas wyboru komponentów nie była brana pod uwagę wodo- oraz kurzoodporność i zakres temperatur pracy. Jednak gdyby słuchawki miały wejść na rynek, musiałyby zostać dodatkowo przystosowane do działania w wymagających warunkach terenowych.