\section{Głośniki}
\label{cha:glosniki}

Każda ze słuchawek została wyposażona w głośnik odpowiedzialny zarówno za doprowadzanie do ucha zwykłych dźwięków z zewnątrz, jak i generowanie antyfazy dla dźwięków niebezpiecznych. Wybrany został głośnik \textit{254-PS604-RO} firmy \textit{Kobitone}. Głównym kryterium wyboru była impedancja głośnika, wynosząca $32\Omega$ oraz moc znamionowa na poziomie $ 200 mW $. Te parametry zapewniały dobrą jakość odtwarzanego dźwięku, która była kluczowa, aby uzyskać maksymalne odwzorowanie dźwięków otoczenia. Celem było to, aby użytkownik czuł się w słuchawkach naturalnie. Choć charakterystyka częstotliwościowa głośnika jest zbliżona do liniowej tylko w zakresie od $ 400 $ do $ 7000 Hz $, to zawiera się w nim większość słyszanych odbieranych przez człowieka dźwięków.

Sterowanie głośnikami zostało przewidziane z wykorzystaniem wbudowanych w mikrokontroler dwóch 12-bitowych przetworników cyfrowo-analogowych w trybie single-ended. Ich wyjścia były dodatkowo wzmacniane przez układy \textit{TPA2005D1DGNR} firmy \textit{Texas Instruments}. Są to wzmacniacze audio klasy D, stworzonej na potrzeby urządzeń przenośnych. Obok innych popularnych klas, jak A, B, AB, czy G, klasa D charakteryzuje się bardzo wysoką wydajnością mocową (nawet powyżej $ 90\% $). Wynika to stąd, że w odróżnieniu od pozostałych, gdzie stosowane są konfiguracje common-emitter lub push-pull, klasa D stosuje całkowite załączanie lub wyłączanie tranzystora wyjściowego i modulację częstotliwości PWM, aby przybliżyć analogowy poziom napięcia\cite{AudioAmps}. Układ \textit{TPA2005D1} ma według specyfikacji wydajność ok. $ 85\% $ przy $32\Omega$ głośniku, zasilaniu$  3.6V $ i mocy wyjściowej $ 100mW $.

Dodatkowo układy te mają wbudowany pin shutdown, który został użyty aby wyłączać je w trybie uśpienia słuchawek opisanym szerzej w rozdziale \textbf{REF DO ROZDZIALU Z SYSTEMEM SŁUCHAWEK}. Dzięki niemu wzmacniacze mogą pobierać zaledwie $ 0.5 µA $.

W idealnym przypadku słuchawki powinny zawierać kilka głośników, co umożliwiłoby symulację dźwięku przestrzennego. To jednak zwiększyłoby koszt słuchawek, wymagało zamontowania kilku mikrofonów, utrudniło rozmieszczenie elementów wewnątrz i implementację algorytmu wyciszającego.