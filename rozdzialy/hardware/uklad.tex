\section{Układ elektroniczny}

Przed przystąpieniem do projektowania właściwego układu elektronicznego konieczne było zadecydowanie, czy wykonać go analogowo, czy cyfrowo. Podejście pierwsze oznaczało użycie inwerterów, wzmacniaczy, filtrów, itp. do uzyskania odpowiednich opóźnienia i fazy dźwięku. Jednak pozostawała wciąż kwestia możliwości zamiennego wyciszania i przekazywania dźwięków w zależności od ich amplitudy. Okazało się, że znalezienie materiałów na ten temat jest wyjątkowo trudne, a szukanie błędów na schemacie mogłoby sprawiać dużo więcej problemów, niż w programie na mikrokontroler. Z tego powodu został wybrany układ cyfrowy, który jak się później okazało, powodował wiele problemów z szumami i opóźnieniami sygnału.

Kolejną decyzją do podjęcia było to, czy każda ze słuchawek będzie miała swój własny układ, czy też jedna będzie odpowiedzialna za obliczenia, a druga jedynie skomunikowana z nią. Zostało wybrane podejście drugie, ponieważ pozwalało to zminimalizować koszty oraz lepiej rozłożyć masę. Jedna słuchawka miała zawierać główną płytkę z mikrokontrolerem i przyciskami, a druga jedynie mikrofon oraz układ ładowania i akumulator. Wadą tego rozwiązania była konieczność równoczesnej analizy dźwięków z obu słuchawek, co przekładało się na szybkość działania mikrokontrolera oraz narażone na szumy przewody prowadzące przez pałąk od jednej słuchawki do drugiej.

Poniżej przedstawiono uproszczony schemat układów dla obu słuchawek.

\textbf{SCHEMAT UKŁADÓW SŁUCHAWEK}

