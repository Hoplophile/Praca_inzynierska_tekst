\chapter{Wstęp}
\label{cha:wstep}

Otaczający nas świat jest pełen dźwięków pochodzących z różnych źródeł. Możemy wyróżnić dźwięki powszechnie uważane za przyjemne dla ucha oraz te nieprzyjemne, z reguły o dużym natężeniu. Narażenie na nadmierny hałas jest w dużej mierze zależne od naszego miejsca pracy. Pracownika biurowego mogą irytować samochody słyszane przez uchylone okno, a dla osoby pracującej na budowie hałas ciężkich maszyn normalny dźwięk otoczenia. Ze względu na charakter pracy, poziom dźwięku w pomieszczeniach biurowych reguluje polska norma "Dopuszczalne wartości poziomu dźwięku na stanowisku pracy"\cite{PolskaNormaCisnienia}. Jednak obaj pracownicy są równie narażeni na skutki przebywania w ciągłym hałasie i potrzebują stosownego zabezpieczenia przed nim.

Wystawienie na zbyt wysoki poziom ciśnienia akustycznego prowadzi do rozdrażnienia, zmęczenia, a w konsekwencji do uszkodzenia słuchu. Z tego powodu tworzone są różne rozwiązania pozwalające polepszyć samopoczucie i chronić słuch. 

Fala akustyczna jest falą mechaniczną, dlatego można ją stosunkowo łatwo tłumić bez użycia elektroniki. Takie rozwiązanie ma jednak swoje wady. Słuchawka musi być duża i ciężka, żeby zmieścić jak najwięcej materiału tłumiącego, a sama metoda działa dobrze dla częstotliwości od $ 20Hz $ do $ 800Hz $ i tłumi do $ 30dB $\cite{SennheiserANC}. Z pomocą przychodzą układy elektroniczne służące do aktywnego tłumienia zakłóceń, czyli nakładania fali przesuniętej w fazie o $180^\circ$ na oryginalny dźwięk. Te z kolei dzielą się na cyfrowe oraz analogowe. 

Praktyczne zastosowania różnych technik tłumienia fal akustycznych są obecne w wielu dziedzinach. Są to między innymi:
\begin{itemize}
	\item przydrożne ekrany dźwiękochłonne
	\item ochronniki słuchu ogólnego zastosowania
	\item słuchawki multimedialne z ANC (ang. \textit{Active Noise Cancelling} - Aktywne Tłumienie Szumu)
	\item systemy wyciszania w pojazdach
	\item strzeleckie ochronniki słuchu
\end{itemize}

W poniższej pracy skupiono się na tych ostatnich. Postanowiono zaprojektować, zaprogramować i zbudować słuchawki taktyczne, które w normalnych warunkach przepuszczają dźwięki z zewnątrz i umożliwiają normalny odbiór dźwięków otoczenia, a w razie wystąpienia niebezpiecznie dużych natężeń - wytłumiają aktywnie, chroniąc słuch.