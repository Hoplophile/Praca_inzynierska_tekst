\chapter{Wykonanie ochronników słuchu}

Głównym założeniem mojej pracy inżynierskiej było wykonanie strzeleckich ochronników słuchu. Zamysł był wzorowany na istniejących produktach, choć miał rozszerzać ich funkcjonalność. Słuchawki miały składać się z materiału tłumiącego, elektronicznego układu przetwarzania dźwięku z otoczenia oraz wyjścia do komunikacji radiowej. Elektroniczny układ spełniał główne zadanie w słuchawkach, ponieważ przekazywał dźwięk z otoczenia oraz z radiotelefonu do ucha użytkownika (aby materiał tłumiący nie zakłócał normalnej komunikacji) oraz dokonywał aktywnego wyciszenia dźwięków (podobnie jak w słuchawkach multimedialnych), kiedy natężenie fali akustycznych przekraczało określony poziom.

Słuchawki miały być stworzone od podstaw aż do otrzymania gotowego produktu, co obejmowało:

\begin{itemize}
    \item dobór parametrów głośników, mikrofonów, baterii i innych elementów
    \item zaprojektowanie, zamówienie i zlutowanie płytki PCB
    \item napisanie oprogramowania do mikrokontrolera przetwarzającego sygnały
    \item zaprojektowanie i wydrukowanie na drukarce 3D obudowy słuchawek
    \item dobór materiału tłumiącego pasywnie
\end{itemize}
