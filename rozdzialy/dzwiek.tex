\chapter{Dźwięk}
\label{cha:dwiek}

\section{Ogólna charakterystyka fal dźwiękowych}
Dźwięk jest wrażeniem słuchowym, powodowanym przez fale akustyczne, które są falami mechanicznymi. W ludzkim uchu ich odbiór odbywa się przez zmiany ciśnienia płynu, którym wypełniony jest ślimak (patrz rys. \ref{ślimak}). Powodują one podrażnianie rzęsek, które z kolei przekazują impulsy elektrochemiczne do mózgu. Im mniejsza częstotliwość fali, tym dalej dociera i na tej podstawie mózg potrafi rozróżniać częstotliwości. Dla człowieka słyszalne są fale z zakresu ok. 20Hz - 20kHz, choć górna granica maleje z wiekiem przez stopniową degradację najbardziej zewnętrznych rzęsek (a więc odpowiadających za najwyższe częstotliwości).\cite{JakSlyszymy} 

\begin{figure}[h]
	\centering
	\includegraphics{zdjecia/ślimak.png}
	\caption{\label{ślimak} Odbiór fal dźwiękowych w ślimaku ludzkiego ucha}
\end{figure}

Dźwięk można opisać kilkoma podstawowymi parametrami:
\begin{itemize}
	\item wysokość (częstotliwość fali)
	\item głośność (amplituda fali)
	\item barwa (skład widmowy)
	\item czas trwania
\end{itemize}

\section{Fale dźwiękowe jako zagrożenie dla zdrowia człowieka}
Głównym, choć nie jedynym, parametrem dźwięku, który determinuje jego szkodliwość jest amplituda fali, czyli ciśnienie akustyczne. W tabeli \ref{tabelaSPL} przedstawiono sytuacje, w których występuje określony poziom natężenia.

\begin{table}[!htb]
	\centering
	\begin{tabular}{|r|l|}
		\hline
		Natężenie [dB] & Sytuacja \\
		\hline
		\hline
		\rowcolor{red!50}
		130 & Młot pneumatyczny \\
		\hline
		\rowcolor{red!40}
		120 & Klakson z odległości 1m \\
		\hline
		\rowcolor{red!30}
		110 & Lotnisko \\
		\hline
		\rowcolor{red!20}
		100 & Przejazd pociągu \\
		\hline
		\rowcolor{orange!20}
		90 & Wnętrze autobusu \\
		\hline
		\rowcolor{yellow!20}
		80 & Zatłoczona ulica \\
		\hline
		\rowcolor{green!30}
		70 & Konwersacja \\
		\hline
		\rowcolor{green!30}
		60 & Salon z cichą muzyką \\
		\hline
		\rowcolor{green!30}
		50 & Biuro \\
		\hline
		\rowcolor{green!30}
		40 & Sypialnia \\
		\hline
		\rowcolor{green!30}
		30 & Studio nagraniowe \\
		\hline
		\rowcolor{green!30}
		20 & Studio radiowe \\
		\hline
		\rowcolor{green!30}
		10 & Próg słyszalności \\
		\hline
	\end{tabular}
	\label{tabelaSPL}
	\caption{Poziomy natężenia dźwięku\cite{SPLtable}}
\end{table}


Amerykańska agencja federalna NIOSH (ang. \textit{National Institute for Occupational Safety and Health}) zajmuje się badaniem i zapobieganiem chorobom związanym z pracą. Wydała ona dokument\cite{NIOSH}, w którym przedstawiono bezpieczny czas wystawienia na określone poziomy ciśnienia akustycznego.

\begin{table}[!htb]
	\centering
	\begin{tabular}{|r|l|}
		\hline
		Natężenie [dB] & Maks. czas [h] \\
		\hline
		\hline
		127 & 00:00:01 \\
		\hline
		118 & 00:00:14 \\
		\hline
		109 & 00:01:53 \\
		\hline
		100 & 00:15:00 \\
		\hline
		91 & 02:00:00 \\
		\hline
		82 & 16:00:00 \\
		\hline
	\end{tabular}
	\label{tabelaSPLczas}
	\caption{Maksymalny czas wystawienia na określone poziomy nateżęnia \cite{SPLtable}}
\end{table}